
%  The AAU Poster Theme.
%  2013-05-08 v. 1.1.0
%  Copyright 2013 by Jesper Kjær Nielsen <jkn@es.aau.dk>
%
%  This is free software: you can redistribute it and/or modify
%  it under the terms of the GNU General Public License as published by
%  the Free Software Foundation, either version 3 of the License, or
%  (at your option) any later version.
%
%  This is distributed in the hope that it will be useful,
%  but WITHOUT ANY WARRANTY; without even the implied warranty of
%  MERCHANTABILITY or FITNESS FOR A PARTICULAR PURPOSE.  See the
%  GNU General Public License for more details.
%
%  You can find the GNU General Public License at <http://www.gnu.org/licenses/>.
\documentclass[a0paper,portrait]{baposter}
\usepackage{kotex}
\usepackage[english]{babel}
\usepackage{helvet}
\renewcommand{\familydefault}{\sfdefault} % for text
\usepackage[helvet]{sfmath} % for math
\usepackage[T1]{fontenc}

\usepackage{graphicx, wrapfig}

\usepackage{caption}
\captionsetup{
  skip=2pt, 
  font=scriptsize,% set font size to scriptsize
  labelfont=bf % bold label (e.g., Figure 3.2) font
}
% Make the standard latex tables look so much better
\usepackage{array,booktabs}
% For creating beautiful plots
\usepackage{pgfplots}
\usetikzlibrary{shapes.geometric}

\usepackage{amsmath}
% Adds new math symbols
\usepackage{amssymb}

% Bibliography package for APA7 format
\usepackage[
  style=apa,
  backend=biber,
  apamaxprtauth=999,
  maxnames=999,
  uniquelist=false
]{biblatex}
\DeclareLanguageMapping{english}{american-apa}
\addbibresource{bibfile.bib}

% Set bibliography font size to tiny
\renewcommand*{\bibfont}{\tiny}

% Set hanging indent for bibliography entries (APA style)
\setlength{\bibhang}{0.1in}  % Reduced hanging indent
\setlength{\bibitemsep}{0.01em}

%%%%%%%%%%%%%%%%%%%%%%%%%%%%%%%%%%%%%%%%%%%%%%%%
% Colours
% http://en.wikibooks.org/wiki/LaTeX/Colors
%%%%%%%%%%%%%%%%%%%%%%%%%%%%%%%%%%%%%%%%%%%%%%%%
\selectcolormodel{RGB}
% define the three aau colors : blue version
\definecolor{aaublue1}{RGB}{33,26,82}% dark blue
\definecolor{aaublue2}{RGB}{113,109,143} % light blue
\definecolor{aaublue3}{RGB}{194,193,204} % lighter blue


%green 
%\definecolor{aaublue1}{RGB}{0,163,136}% dark green
%\definecolor{aaublue2}{RGB}{121,189,143} % light green
%\definecolor{aaublue3}{RGB}{190,235,159} % lighter green
%\definecolor{aaublue4}{RGB}{255,255,157} % yellow
%\definecolor{aaublue5}{RGB}{255,97,56}



%%%%%%%%%%%%%%%%%%%%%%%%%%%%%%%%%%%%%%%%%%%%%%%%
% Lists
% http://en.wikibooks.org/wiki/LaTeX/List_Structures
%%%%%%%%%%%%%%%%%%%%%%%%%%%%%%%%%%%%%%%%%%%%%%%%
% Easier configuration of lists
\usepackage{enumitem}
%configure itemize
\setlist{%
  topsep=0pt,% set space before and after list
  noitemsep,% remove space between items
  labelindent=\parindent,% set the label indentation to the paragraph indentation
  leftmargin=*,% remove the left margin
  font=\color{aaublue1}\normalfont, %set the colour of all bullets, numbers and descriptions to aaublue1
}
% use set<itemize,enumerate,description> if you have an older latex distribution
\setitemize[1]{label={\raise1.25pt\hbox{$\blacktriangleright$}}}
\setitemize[2]{label={\scriptsize\raise1.25pt\hbox{$\blacktriangleright$}}}
\setitemize[3]{label={\raise1.25pt\hbox{$\star$}}}
\setitemize[4]{label={-}}
%\setenumerate[1]{label={\theenumi.}}
%\setenumerate[2]{label={(\theenumii)}}
%\setenumerate[3]{label={\theenumiii.}}
%\setenumerate[4]{label={\theenumiv.}}
%\setdescription{font=\color{aaublue1}\normalfont\bfseries}

% use setlist[<itemize,enumerate,description>,<level>] if you have a newer latex distribution
%\setlist[itemize,1]{label={\raise1.25pt\hbox{$\blacktriangleright$}}}
%\setlist[itemize,2]{label={\scriptsize\raise1.25pt\hbox{$\blacktriangleright$}}}
%\setlist[itemize,3]{label={\raise1.25pt\hbox{$\star$}}}
%\setlist[itemize,4]{label={-}}
%\setlist[enumerate,1]{label={\theenumi.}}
%\setlist[enumerate,2]{label={(\theenumii)}}
%\setlist[enumerate,3]{label={\theenumiii.}}
%\setlist[enumerate,4]{label={\theenumiv.}}
%\setlist[description]{font=\color{aaublue1}\normalfont\bfseries}

%%%%%%%%%%%%%%%%%%%%%%%%%%%%%%%%%%%%%%%%%%%%%%%%
% Misc
%%%%%%%%%%%%%%%%%%%%%%%%%%%%%%%%%%%%%%%%%%%%%%%%
% change/remove some names
\addto{\captionsenglish}{
  %remove the title of the bibliograhpy
  \renewcommand{\refname}{\vspace{-0.7em}}
  %change Figure to Fig. in figure captions
  \renewcommand{\figurename}{Fig.}
}
% create links
\usepackage{url}
%note that the hyperref package is currently incompatible with the baposter class

%%%%%%%%%%%%%%%%%%%%%%%%%%%%%%%%%%%%%%%%%%%%%%%%
% Macros
%%%%%%%%%%%%%%%%%%%%%%%%%%%%%%%%%%%%%%%%%%%%%%%%
\newcommand{\alert}[1]{{\color{aaublue1}#1}}

%%%%%%%%%%%%%%%%%%%%%%%%%%%%%%%%%%%%%%%%%%%%%%%%
% Document Start 
%%%%%%%%%%%%%%%%%%%%%%%%%%%%%%%%%%%%%%%%%%%%%%%%
\begin{document}
%%%%%%%%%%%%%%%%%%%%%%%%%%%%%%%%%%%%%%%%%%%%%%%%
% Some changes that cannot be made in the preamble
%%%%%%%%%%%%%%%%%%%%%%%%%%%%%%%%%%%%%%%%%%%%%%%%
% set the background of the poster
\background{
  \begin{tikzpicture}[remember picture,overlay]%
    %the poster background color
    \fill[fill=aaublue3] (current page.north west) rectangle (current page.south east);
    %the header
    \fill [fill=white] (current page.north west) rectangle ([yshift=-\headerheight] current page.north east);
  \end{tikzpicture}
}
% if you want to reduce the space before and after equations, use and adjust
% the following lines
%\addtolength{\abovedisplayskip}{-2mm}
%\addtolength{\belowdisplayskip}{-2mm}

%%%%%%%%%%%%%%%%%%%%%%%%%%%%%%%%%%%%%%%%%%%%%%%%
% General poster setup
%%%%%%%%%%%%%%%%%%%%%%%%%%%%%%%%%%%%%%%%%%%%%%%%
\begin{poster}{
  %general options for the poster
  grid=false,
  columns=3,
%  colspacing=4.2mm,
  headerheight=0.1\textheight,
  background=user,
%  bgColorOne=red!42, %is used when background != user and none
%  bgColortwo=green!42, %is used when background is shaded
  eyecatcher=true,
  %posterbox options
  headerborder=closed,
  borderColor=aaublue1,
  headershape=rectangle,
  headershade=plain,
  headerColorOne=white,
%  headerColortwo=yellow!42, %is used when the header background is shaded
  textborder=rectangle,
  boxshade=plain,
  boxColorOne=white,
%  boxColorTwo=cyan!42,%is used when the text background is shaded
  headerFontColor=black,
  headerfont=\Large\sf\bf,
  linewidth=1pt
}
%the Eye Catcher (the logo on the left)
{
  %this can be commented out or replaced by a company/department logo
  \includegraphics[height=0.75\headerheight]{./logo/Gemini_Generated_Image_a5kem9a5kem9a5ke.png}
}
%the poster title
{\fontsize{18}{60}\selectfont
\color{black}\bf
  Enhancing Science Data Literacy Through Interactive Analysis of the Keeling Curve: A Case Study with Science-Gifted Students
  % 킬링 곡선의 상호작용적 분석을 통한 과학 데이터 리터러시 함양 : 과학영재학생들에게 적용
}
%the author(s)
{\color{black}\small
  \vspace{1em} Kiehyun Park\textsuperscript{1} (Kiehyun.Park@gmail.com)  \textsuperscript{1}Korea National University of Education
  % 박기현, 한국교원대학교
   }
%the logo (the logo on the right)
{
  %this can be commented out or replaced by a company/department logo
  \includegraphics[height=0.75\headerheight]{./logo/Knue_logo.png}
}

%%%%%%%%%%%%%%%%%%%%%%%%%%%%%%%%%%%%%%%%%%%%%%%%
% the actual content of the poster begins here
%%%%%%%%%%%%%%%%%%%%%%%%%%%%%%%%%%%%%%%%%%%%%%%%

\begin{posterbox}[name=intro,column=0,row=0,span=2]{Abstract}
\footnotesize
Climate change education requires students to develop data literacy skills to critically evaluate evidence. 
% 기후변화 교육은 학생들이 증거를 비판적으로 평가하기 위한 데이터 리터러시 기술을 개발할 것을 요구한다.
This study presents a Python-based educational approach using authentic atmospheric CO\textsubscript{2} data from the Keeling Curve to develop Earth science data literacy among secondary school students. 
% 본 연구는 킬링 곡선의 실제 대기 CO2 데이터를 사용하는 파이썬 기반 교육 접근법을 제시하여 중등학교 학생들의 지구과학 데이터 리터러시를 개발한다.
Students engage with 67 years of continuous measurements from Mauna Loa Observatory, learning to access authoritative data sources, apply computational tools for visualization and analysis, and understand the relationship between data and scientific conclusions about climate change. 
% 학생들은 마우나 로아 관측소의 67년간 연속 측정 데이터를 다루면서, 권위 있는 데이터 출처에 접근하고, 시각화 및 분석을 위한 계산 도구를 적용하며, 데이터와 기후변화에 대한 과학적 결론 간의 관계를 이해하는 것을 학습한다.
Assessment results demonstrate significant improvements in students' ability to interpret scientific data and understand measurement uncertainty \parencite{Kjelvik2019,Gould2016}.
% 평가 결과는 학생들의 과학 데이터 해석 능력과 측정 불확실성 이해에서 유의미한 향상을 보여준다.
\end{posterbox}

\begin{posterbox}[name=usage,column=0,below=intro]{Introduction}
\footnotesize
\textbf{Background:} 
% 배경:
The Keeling Curve, documenting atmospheric CO\textsubscript{2} concentrations since 1958, provides one of the most powerful datasets for climate change education \parencite{Keeling1960,Keeling1976}. 
% 1958년 이래 대기 CO2 농도를 기록한 킬링 곡선은 기후변화 교육을 위한 가장 강력한 데이터셋 중 하나를 제공한다.
However, many students lack the data literacy skills necessary to critically engage with such scientific evidence.
% 그러나, 많은 학생들은 이러한 과학적 증거와 비판적으로 상호작용하는 데 필요한 데이터 리터러시 기술이 부족하다.
\vspace{-0.5em}
\begin{center}
\includegraphics[width=0.85\linewidth]{images/06_keeling_curve_trend.pdf}
\captionof{figure}{Example of Keeling Curve visualization showing seasonal variation and long-term trend}
% 그림: 계절적 변동과 장기 추세를 보여주는 킬링 곡선 시각화 예시
\label{fig:keeling}
\end{center}
\vspace{-0.5em}
\textbf{Why Data Literacy Matters:} 
% 데이터 리터러시가 중요한 이유:
Data literacy—the ability to read, work with, analyze, and argue with data \parencite{Wolff2016}—is essential for understanding climate science. 
% 데이터 리터러시—데이터를 읽고, 다루고, 분석하고, 논증하는 능력—는 기후 과학을 이해하는 데 필수적이다.
Students must move beyond passive consumption to actively engage with primary data sources \parencite{NRC2012}.
% 학생들은 수동적 소비를 넘어 1차 데이터 출처와 능동적으로 상호작용해야 한다.
\vspace{0.5em}

\textbf{Role of Computation:} 
% 계산의 역할:
Python and Jupyter notebooks provide accessible tools for students to engage with authentic scientific datasets, developing both conceptual understanding and practical skills \parencite{Barba2019,Weintrop2016}.
% 파이썬과 주피터 노트북은 학생들이 실제 과학 데이터셋과 상호작용할 수 있는 접근 가능한 도구를 제공하여, 개념적 이해와 실용적 기술을 모두 개발한다.

\end{posterbox}

\begin{posterbox}[name=lists,column=0,below=usage, above=bottom]{Educational Framework}
% 교육 프레임워크

\footnotesize \textbf{1. Learning Objectives} 
% 1. 학습 목표
\scriptsize
Students will be able to:
% 학생들은 다음을 할 수 있게 된다:
\begin{itemize}
\item Access and evaluate authoritative scientific data sources
% 권위 있는 과학 데이터 출처에 접근하고 평가하기
\item Apply Python tools for data visualization and analysis
% 데이터 시각화와 분석을 위한 파이썬 도구 적용하기
\item Interpret temporal patterns in atmospheric CO\textsubscript{2} data
% 대기 CO2 데이터의 시간적 패턴 해석하기
\item Understand the relationship between data and climate science conclusions
% 데이터와 기후 과학 결론 간의 관계 이해하기
\end{itemize}
\vspace{0.5em}
\footnotesize \textbf{2. The 5E Instructional Model} 
% 2. 5E 교수 모형
\scriptsize
This module follows the evidence-based 5E framework \parencite{Bybee2006}:
% 본 모듈은 증거 기반 5E 프레임워크를 따른다:

% \textbf{Engage:} Introduction to Keeling Curve history
% % 참여: 킬링 곡선 역사 소개
% \textbf{Explore:} Hands-on data download and visualization
% % 탐구: 실습으로 데이터 다운로드 및 시각화
% \textbf{Explain:} Understanding patterns and trends
% % 설명: 패턴과 추세 이해하기
% \textbf{Elaborate:} Advanced analysis and calculations
% % 정교화: 고급 분석 및 계산
% \textbf{Evaluate:} Student-designed investigations
% % 평가: 학생 설계 탐구

\vspace{-1.0em}
\begin{center}
\resizebox{0.75\textwidth}{!}{
\begin{tikzpicture}[every node/.style={font=\scriptsize}]
  % Define colors for each phase
  \definecolor{engage}{RGB}{255,182,193}
  \definecolor{explore}{RGB}{173,216,230}
  \definecolor{explain}{RGB}{255,218,185}
  \definecolor{elaborate}{RGB}{221,160,221}
  \definecolor{evaluate}{RGB}{152,251,152}
  
  % Pentagon vertices (radius = 2.5cm, starting from top)
  % 1. Engage - top (90 degrees)
  \node[draw, ellipse, fill=engage, minimum width=2.0cm, minimum height=1.4cm, thick, text width=1.8cm, align=center, inner sep=0.05cm] (engage) at (0, 2.5) {
    \textbf{\textcolor{red!60!black}{1. Engage}}
    Introduction to Keeling Curve history
  };
  
  % 2. Explore - upper left (162 degrees)
  \node[draw, ellipse, fill=explore, minimum width=2.0cm, minimum height=1.4cm, thick, text width=1.8cm, align=center, inner sep=0.05cm] (explore) at (-3.0, 1) {
    \textbf{\textcolor{blue!60!black}{2. Explore}}
    Hands-on data download and visualization
  };
  
  % 3. Explain - lower left (234 degrees)
  \node[draw, ellipse, fill=explain, minimum width=2.0cm, minimum height=1.4cm, thick, text width=1.8cm, align=center, inner sep=0.05cm] (explain) at (-1.8, -1.1) {
    \textbf{\textcolor{orange!60!black}{3. Explain}}
    Understanding patterns and trends
  };
  
  % 4. Elaborate - lower right (306 degrees)
  \node[draw, ellipse, fill=elaborate, minimum width=2.0cm, minimum height=1.4cm, thick, text width=1.8cm, align=center, inner sep=0.05cm] (elaborate) at (1.8, -1.1) {
    \textbf{\textcolor{purple!60!black}{4. Elaborate}}
    Advanced analysis and calculations
  };
  
  % 5. Evaluate - upper right (18 degrees)
  \node[draw, ellipse, fill=evaluate, minimum width=2.0cm, minimum height=1.4cm, thick, text width=1.8cm, align=center, inner sep=0.05cm] (evaluate) at (3.0, 1) {
    \textbf{\textcolor{green!50!black}{5. Evaluate}}
    Student-designed investigations
  };
  
  % Central label
  % \node[font=\scriptsize\bfseries, text=aaublue1] at (0, 0) {5E Cycle};
  
  % Arrows connecting the phases in cycle
  \draw[->, very thick, red!70!black] (engage) -- (explore);
  \draw[->, very thick, blue!70!black] (explore) -- (explain);
  \draw[->, very thick, orange!70!black] (explain) -- (elaborate);
  \draw[->, very thick, purple!70!black] (elaborate) -- (evaluate);
  % \draw[->, very thick, green!70!black] (evaluate) -- (engage);
  
\end{tikzpicture}
}
\end{center}
\captionof{figure}{The 5E instructional model used in this module}

\vspace{0.2em}
\footnotesize \textbf{3. Alignment with Standards} 
% 3. 표준과의 정렬
\scriptsize
This approach aligns with Next Generation Science Standards (NGSS), emphasizing science practices, particularly analyzing and interpreting data \parencite{NGSS2013,NRC2012}.
% 이 접근법은 차세대 과학 표준(NGSS)과 일치하며, 특히 데이터 분석 및 해석의 과학적 실천을 강조한다.

\vspace{0.5em}
\footnotesize \textbf{4. The Keeling Curve Dataset} 
% 4. 킬링 곡선 데이터셋
\scriptsize
The Scripps CO\textsubscript{2} Program's continuous atmospheric measurements provide the longest high-precision record of atmospheric carbon dioxide, serving as the foundation for understanding anthropogenic climate change. 
% 스크립스 CO2 프로그램의 연속적인 대기 측정은 대기 중 이산화탄소의 가장 긴 고정밀도 기록을 제공하며, 인위적 기후변화 이해의 기반이 된다.
This dataset combines exceptional temporal coverage with rigorous quality control, making it ideal for educational applications.
% 이 데이터셋은 뛰어난 시간적 범위와 엄격한 품질 관리를 결합하여 교육적 활용에 이상적이다.

\begin{itemize}
\item \textbf{Source:} Scripps CO\textsubscript{2} Program, Mauna Loa Observatory
% 출처: 스크립스 CO2 프로그램, 마우나 로아 관측소
\item \textbf{Duration:} 1958-present (67 years continuous)
% 기간: 1958년-현재 (67년간 연속)
\item \textbf{Significance:} Shows both seasonal variation and long-term trend \parencite{Keeling1976,Thoning1989}
% 중요성: 계절적 변동과 장기 추세를 모두 표시
\item \textbf{Current level:} 422.5 ppm (52\% above pre-industrial)
% 현재 수준: 422.5 ppm (산업화 이전보다 52% 높음)
\end{itemize}    


\end{posterbox}

\begin{posterbox}[name=Conclusion,column=1,below=intro]{Methodology}
% 방법론
    \footnotesize
    \textbf{Technical Implementation}
    % 기술적 구현
    \scriptsize
    \begin{itemize}
    \item \textbf{Platform:} Jupyter Notebook with Python 3.8+
    % 플랫폼: 파이썬 3.8+ 기반 주피터 노트북
    \item \textbf{Libraries:} pandas, matplotlib, numpy
    % 라이브러리: pandas, matplotlib, numpy
    \item \textbf{Cloud Option:} Google Colab for easy access
    % 클라우드 옵션: 쉽운 접근을 위한 Google Colab
    \item \textbf{Data Source:} Scripps CO\textsubscript{2} Program (scrippsco2.ucsd.edu)
    % 데이터 출처: 스크립스 CO2 프로그램
    \end{itemize}
    
    \vspace{0.2em}
    \footnotesize
    \textbf{Student Learning Activities} \\
    % 학생 학습 활동
    \vspace{-1.8em}
    \begin{center}
    \resizebox{0.9\textwidth}{!}{
    \begin{tikzpicture}[every node/.style={font=\scriptsize}]
      % Define colors
      \definecolor{activity1}{RGB}{238,130,130}
      \definecolor{activity2}{RGB}{130,200,238}
      \definecolor{activity3}{RGB}{144,238,144}
      \definecolor{activity4}{RGB}{255,218,130}
      
      % Activity 1 (Top Left)
      \node[draw, rounded corners, fill=activity1, minimum width=4.2cm, minimum height=2.2cm, thick, text width=3.8cm, align=left, inner sep=0.15cm] (act1) at (0,0) {
        \textbf{\textcolor{red!50!black}{Activity 1: Data Acquisition}} \\[0.2em]
        Students download real CO\textsubscript{2} data from \textcolor{red!70!black}{Mauna Loa Observatory}, examine file structure and metadata, and understand data quality indicators.
      };
      
      % Activity 2 (Top Right)
      \node[draw, rounded corners, fill=activity2, minimum width=4.2cm, minimum height=2.2cm, thick, text width=3.8cm, align=left, inner sep=0.15cm] (act2) at (4.6,0) {
        \textbf{\textcolor{blue!50!black}{Activity 2: Basic Visualization}} \\[0.2em]
        Create time series plots to identify the annual \textcolor{blue!70!black}{``breathing'' pattern} caused by Northern Hemisphere vegetation cycles.
      };
      
      % Activity 4 (Bottom Left)
      \node[draw, rounded corners, fill=activity4, minimum width=4.2cm, minimum height=2.2cm, thick, text width=3.8cm, align=left, inner sep=0.15cm] (act4) at (0,-2.6) {
        \textbf{\textcolor{orange!50!black}{Activity 4: Global Comparison}} \\[0.2em]
        Compare measurements from multiple stations (\textcolor{orange!70!black}{South Pole, Alaska, Hawaii}) to understand spatial variability.
      };
      
      % Activity 3 (Bottom Right)
      \node[draw, rounded corners, fill=activity3, minimum width=4.2cm, minimum height=2.2cm, thick, text width=3.8cm, align=left, inner sep=0.15cm] (act3) at (4.6,-2.6) {
        \textbf{\textcolor{green!40!black}{Activity 3: Trend Analysis}} \\[0.2em]
        Calculate \textcolor{green!50!black}{rate of change} across different time periods: \\
        \textcolor{green!50!black}{1960-1990:} 1.3 ppm/year \\
        \textcolor{green!50!black}{1990-2020:} 2.0 ppm/year \\
        \textcolor{green!50!black}{2010-2024:} 2.5 ppm/year
      };
      
      % Arrows
      \draw[->, very thick, blue!60!black] (act1) -- (act2);
      \draw[->, very thick, green!60!black] (act2) -- (act3);
      \draw[->, very thick, orange!60!black] (act3) -- (act4);
      
    \end{tikzpicture}
    }
    
    \end{center}
    \captionof{figure}{Student learning activities aligned to the 5E cycle}

\vspace{-1.0em}
\begin{center}
\includegraphics[trim=0 0 0 50, clip, width=0.8\linewidth]{images/seasonal_variation_2000_2002.pdf}
\captionof{figure}{Keeling Curve: Seasonal Variation Detail (2000-2002) Understanding Earth\'s "Breathing" Pattern}
\label{fig:seasonal variation}
\end{center}

\vspace{-1.0em}
\begin{center}
\includegraphics[trim=0 0 0 50, clip, width=0.8\linewidth]{images/co2_three_stations_combined.pdf}
\captionof{figure}{Comparison of atmospheric CO\textsubscript{2} across Mauna Loa, Point Barrow, and the South Pole}
\label{fig:3stations}
\end{center}

\end{posterbox}


\begin{posterbox}[name=figures,column=1,below=Conclusion, above=bottom]{Enhance Science Data Literacy}
% 데이터 리터러시 함양

\footnotesize
    \textbf{Critical Thinking Questions}
    % 비판적 사고 질문
    \scriptsize
    Throughout the activities, students engage with questions that develop analytical thinking \parencite{Chinn2002}:
    % 활동 전반에 걸쳐, 학생들은 분석적 사고를 개발하는 질문에 참여한다:
    
    \begin{itemize}
    \item Why was Mauna Loa chosen as the measurement location?
    % 왜 마우나 로아가 측정 장소로 선택되었는가?
    \item What causes the annual ``breathing'' pattern in CO\textsubscript{2}?
    % CO2의 연간 "호흡" 패턴의 원인은 무엇인가?
    \item How does the rate of increase compare across decades?
    % 수십 년간 증가율은 어떻게 비교되는가?
    \item How can we distinguish natural variation from human-caused trends?
    % 자연적 변동과 인간 유발 추세를 어떻게 구별할 수 있는가?
    \end{itemize}
    
    \vspace{0.5em}
    \footnotesize
    \textbf{Key Science Data Literacy Skills Developed}
    % 개발된 핵심 데이터 리터러시 기술
    \scriptsize
    \begin{enumerate}
    \item \textbf{Data Source Evaluation}
    % 데이터 출처 평가
    \begin{itemize}
    \item Verifying data provenance (Scripps Institution)
    % 데이터 출처 확인 (스크립스 연구소)
    \item Understanding measurement methodology
    % 측정 방법론 이해
    \item Recognizing authoritative vs unreliable sources
    % 권위 있는 출처와 신뢰할 수 없는 출처 구별
    \end{itemize}
    
    \item \textbf{Data Quality Understanding}
    % 데이터 품질 이해
    \begin{itemize}
    \item Handling missing data
    % 결측값 처리
    \item Understanding measurement precision ($\pm$0.3 ppm)
    % 측정 정밀도 이해 (±0.3 ppm)
    \item Recognizing quality control procedures
    % 품질 관리 절차 인식
    \end{itemize}
    
    \item \textbf{Statistical Interpretation}
    % 통계적 해석
    \begin{itemize}
    \item Distinguishing trend from variation
    % 추세와 변동 구별
    \item Calculating rates of change
    % 변화율 계산
    \item Understanding seasonal adjustment
    % 계절 조정 이해
    \end{itemize}
    
    \item \textbf{Computational Skills}
    % 계산 기술
    \begin{itemize}
    \item Loading and processing CSV data with pandas
    % pandas를 사용한 CSV 데이터 로드 및 처리
    \item Creating publication-quality visualizations
    % 출판 품질의 시각화 생성
    \item Writing Python code for analysis
    % 분석을 위한 파이썬 코드 작성
    \end{itemize}
    \end{enumerate}
    
    \vspace{0.5em}
    \footnotesize
    \textbf{Real-World Connections}
    % 실제 세계와의 연결
    \scriptsize
    Students contextualize their findings \parencite{Monroe2019}:
    % 학생들은 자신의 발견을 맥락화한다:
    \begin{itemize}
    \item Pre-industrial CO\textsubscript{2}: 280 ppm
    % 산업화 이전 CO2: 280 ppm
    \item Start of Keeling measurements (1958): 315 ppm
    % 킬링 측정 시작 (1958): 315 ppm
    \item Current level (2024): 422.5 ppm
    % 현재 수준 (2024): 422.5 ppm
    \item \textbf{52\% increase since industrial revolution}
    % 산업혁명 이후 52% 증가
    \end{itemize}

\end{posterbox}



\begin{posterbox}[name=install,span=1,column=2]{Results}
% 결과

    \footnotesize \textbf{Student Learning Outcomes} 
    % 학생 학습 성과
    \scriptsize 
    Assessment of 36 science-gifted high school 12th-grade students who completed the module showed significant improvements across multiple dimensions \parencite{Pellegrino2013}.
    % 모듈을 완료한 36명의 과학영재학교 3학년(12학년)에 대한 평가는 여러 차원에서 유의미한 향상을 보여주었다.
    
    \vspace{0.5em}
    \footnotesize
    \textbf{Conceptual Understanding:}
    % 개념적 이해:
    \scriptsize
    \begin{itemize}
    \item Correctly explained seasonal CO\textsubscript{2} oscillations
    % 계절적 CO2 진동을 정확하게 설명
    \item Identified the long-term increasing trend
    % 장기 증가 추세를 식별
    \item Successfully calculated rate of change across decades
    % 수십 년간의 변화율을 성공적으로 계산
    \item Explained why Mauna Loa is appropriate for measurements
    % 마우나 로아가 측정에 적합한 이유를 설명
    \end{itemize}
    
    \vspace{0.5em}
    \footnotesize
    \textbf{Data Literacy Skills:}
    % 데이터 리터러시 기술:
    \scriptsize
    \begin{itemize}
    \item Demonstrated ability to access authoritative data sources
    % 권위 있는 데이터 출처에 접근하는 능력을 시연
    \item correctly interpreted data quality indicators
    % 데이터 품질 지표를 정확하게 해석
    \item distinguished raw vs seasonally-adjusted data
    % 원시 데이터와 계절 조정 데이터를 구별
    \item created appropriate visualizations
    % 적절한 시각화를 생성
    \end{itemize}
    
    \vspace{0.5em}
    \footnotesize
    \textbf{Computational Competency:}
    % 계산 역량:
    \scriptsize
    Students successfully executed Python code, modified code to answer new questions, wrote original code for calculations, and understood DataFrame operations.
    % 파이썬 코드를 성공적으로 실행하고, 새로운 질문에 답하기 위해 코드를 수정하며, 계산을 위한 독창적인 코드를 작성하고 DataFrame 연산을 이해
    
    % \centering
    % \includegraphics[trim=0 0 0 0, clip, width=0.9\linewidth]{images/06_keeling_curve_trend.png}
    % \captionof{figure}{Student-created visualization showing both seasonal variation and 67-year increasing trend in atmospheric CO\textsubscript{2}}
    % \label{fig:student_work}
   
    \vspace{0.5em}
    \footnotesize
    \textbf{Engagement Metrics:}
    % 참여도 지표:
    \scriptsize
    \begin{itemize}
    \item increase in interest in climate science (pre/post survey)
    % 기후 과학에 대한 관심 증가 (사전/사후 설문)
    \item rated activity as ``interesting'' or ``very interesting''
    % 활동을 "흥미롭다" 또는 "매우 흥미롭다"로 평가
    \item preferred real data vs textbook examples
    % 교과서 예시보다 실제 데이터를 선호
    \item expressed interest in using Python for future projects
    % 향후 프로젝트에 파이썬 사용에 관심 표명
    \end{itemize}

\end{posterbox}

\begin{posterbox}[name=equation, column=2, below=install]{Discussion \& Conclusions}
% 토의 및 결론

\footnotesize \textbf{Strengths of the Approach}
% 접근법의 강점
\scriptsize
\begin{itemize}
\item \textbf{Authentic Science:} Using real data from authoritative sources (Scripps Institution) provides genuine scientific experience \parencite{Kjelvik2019}
% 진정한 과학: 권위 있는 출처(스크립스 연구소)의 실제 데이터 사용은 진정한 과학적 경험을 제공
\item \textbf{Scaffolded Learning:} Progressive complexity allows diverse learners to engage at appropriate levels
% 비계화된 학습: 점진적 복잡성은 다양한 학습자가 적절한 수준에서 참여할 수 있게 함
\item \textbf{Interdisciplinary:} Integrates Earth science, mathematics, computer science, and communication
% 학제간 통합: 지구과학, 수학, 컴퓨터과학, 커뮤니케이션 통합
\item \textbf{Evidence-Based:} Students see climate change in data, not just rhetoric \parencite{IPCC2021}
% 증거 기반: 학생들은 단순한 수사가 아닌 데이터에서 기후변화를 보게 됨
\end{itemize}

\vspace{0.5em}
\footnotesize \textbf{Challenges and Solutions}
% 도전 과제와 해결책
\scriptsize
\begin{itemize}
\item \textit{Challenge:} Limited programming experience \\
% 도전: 제한된 프로그래밍 경험
\textit{Solution:} Provide pre-written code cells; use cloud platforms (Google Colab)
% 해결책: 미리 작성된 코드 셀 제공; 클라우드 플랫폼 (Google Colab) 사용
\item \textit{Challenge:} Time constraints \\
% 도전: 시간 제약
\textit{Solution:} Modular design allows flexible implementation
% 해결책: 모듈식 설계로 유연한 구현 가능
\item \textit{Challenge:} Mathematical prerequisites \\
% 도전: 수학적 선수 지식
\textit{Solution:} Visual explanations before formal treatment
% 해결책: 형식적 처리 전에 시각적 설명 제공
\end{itemize}

\vspace{0.5em}
\footnotesize \textbf{Key Findings}
% 핵심 발견
\scriptsize
\begin{enumerate}
\item Students successfully engage with authentic scientific datasets using computational tools \parencite{Weintrop2016}
% 학생들은 계산 도구를 사용하여 실제 과학 데이터셋과 성공적으로 상호작용함
\item Direct data interaction significantly enhances understanding of climate evidence
% 직접적인 데이터 상호작용은 기후 증거에 대한 이해를 크게 향상시킴
\item Data literacy skills are transferable to other scientific contexts
% 데이터 리터러시 기술은 다른 과학적 맥락으로 전이 가능
\item Aligns with NGSS standards for science practices \parencite{NGSS2013}
% NGSS 과학 실천 표준과 일치
\end{enumerate}

\vspace{0.5em}
\footnotesize \textbf{Broader Significance}
% 폭넓은 의미
\scriptsize
This approach demonstrates that climate change education can move beyond passive information consumption to active scientific inquiry. 
% 이 접근법은 기후변화 교육이 수동적 정보 소비를 넘어 능동적 과학 탐구로 나아갈 수 있음을 보여줌.
By developing data literacy through the iconic Keeling Curve dataset, students gain both conceptual understanding and practical skills essential for 21st-century citizenship \parencite{Monroe2019}.
% 상징적인 킬링 곡선 데이터셋을 통해 데이터 리터러시를 개발함으로써, 학생들은 21세기 시민성에 필수적인 개념적 이해와 실용적 기술을 모두 얻게 됨.

\vspace{0.5em}
\footnotesize \textbf{Future Directions}
% 향후 방향
\scriptsize
\begin{itemize}
\item Integration with temperature, ocean, and ice datasets
% 기온, 해양, 빙하 데이터셋과의 통합
\item Machine learning applications for predictive modeling
% 예측 모델링을 위한 기계학습 응용
\item Web-based interfaces for broader accessibility
% 더 넓은 접근성을 위한 웹 기반 인터페이스
\item Cross-cultural adaptation and translation
% 문화간 적응 및 번역
\end{itemize}

\end{posterbox}

\begin{posterbox}[name=refs,column=2,below=equation,above=bottom]{References}
% 참고문헌
% In the last box, you will usually have a list of references
% The bibliography automatically adds the title "References", but
% this have been removed in the preamble

\tiny
\printbibliography[heading=none] 

\vspace{1em}

\vspace{0.5em}
\textbf{Data Availability:} 
% 데이터 가용성:
All Jupyter notebooks and educational materials are available at: \\
% 모든 주피터 노트북과 교육 자료는 다음에서 이용 가능:
\texttt{github.com/[repository-link]}

\vspace{0.5em}

\tiny
Atmospheric CO\textsubscript{2} data courtesy of Scripps CO\textsubscript{2} Program, Scripps Institution of Oceanography, UC San Diego (\url{https://scrippsco2.ucsd.edu/}).
% 대기 CO2 데이터는 UC San Diego 소속 Scripps 해양학 연구소의 Scripps CO2 프로그램 제공

\end{posterbox}

\end{poster}
\end{document}
